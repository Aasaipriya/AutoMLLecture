\renewcommand{\vec}[1]{\mathbf{#1}}
\newcommand{\lecture}{AutoML}
\newcommand{\lecturelong}{Automated Machine Learning}
\newcommand{\semester}{SS 2019}
\newcommand{\assignment}[1]{\nth{#1} Assignment}
\newcommand{\lectors}{M. Lindauer \& F. Hutter}
\newcommand{\hide}[1]{}


\newcommand{\gccs}{\paragraph{General constraints for code submissions}
{
\footnotesize{
Please adhere to these rules to make our and your life easier! We will deduct points if your solution does not fulfill the following:
    
    \begin{itemize}
    	\item If not stated otherwise, we will use exclusively Python $3.5$.
        \item If not stated otherwise, we expect a Python script, which we will invoke exactly as stated on the exercise sheet.
        \item Your solution exactly returns the required output (neither less nor more) -- you can implement a \texttt{--verbose} option to increase the verbosity level for developing.
        \item Add comments and docstrings, so we can understand your solution.
        \item (If applicable) The \texttt{README} describes how to install requirements or provides addition information.
        \item (If applicable) Add required additional packages to \texttt{requirements.txt}. Explain in your \texttt{README} what this package does, why you use that package and provide a link to it's documentation or GitHub page.
        \item (If applicable) All prepared unittests have to pass.
        \item (If applicable) You can (and sometimes have to) reuse code from previous exercises.
    \end{itemize}
    \rule{\textwidth}{.5pt}
    \smallskip\\
    \noindent}}}
