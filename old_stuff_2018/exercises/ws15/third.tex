\documentclass{exam}
\usepackage{amsmath, amsfonts}
\usepackage{verbatim}
\usepackage{graphicx}


\usepackage[hyperfootnotes=false]{hyperref}

\usepackage[usenames,dvipsnames]{color}
\newcommand{\note}[1]{
	\noindent~\\
	\vspace{0.25cm}
	\fcolorbox{Red}{Orange}{\parbox{0.99\textwidth}{#1\\}}
	%{\parbox{0.99\textwidth}{#1\\}}
	\vspace{0.25cm}
}
\renewcommand{\note}[1]{}

\newcommand{\hide}[1]{}

\renewcommand{\vec}[1]{\mathbf{#1}}
\DeclareMathOperator*{\argmin}{argmin}

\qformat{\thequestion. \textbf{\thequestiontitle}\hfill[\thepoints]}
\bonusqformat{\thequestion. \textbf{\thequestiontitle}\hfill[\thepoints]}

\pagestyle{headandfoot}
\firstpageheader{Due: 16.11.2015 (23:59 GMT)}{ {\bf MLOAD} \\ Third Assignment}{M. Lindauer \& F. Hutter\\ WS 2015/16}
\runningheader{Due: 16.11.2015 (23:59 GMT)}{Third Assignment}{WS 2015/16}
\runningfooter{}{}{}
\headrule
\pointsinrightmargin
\bracketedpoints
\marginpointname{pt.}


\begin{document}

After you now know some examples of hard combinatorial problems, the goal of this third exercise is to let you implement some simple Python programs to solve CSP and SAT problem with SLS.


\begin{questions}


	\titledquestion{Solving SAT with uninformed random work (URW)}[15]
	
	Consider the model-finding variant of SAT, i.e., finding a satisfying assignment of a given Boolean formula in CNF (conjunctive normal form).
	A CNF is given in the DIMACS format\footnote{\url{http://www.satcompetition.org/2009/format-benchmarks2009.html}} as follows:

\begin{verbatim}
c start with comments
p cnf 5 3
1 -5 4 0
-1 5 3 4 0
-3 -4 0
\end{verbatim}

\begin{itemize}
  \item Any line starting with a \texttt{c} is a comment;
  \item The single line starting with \texttt{p} defines the number of variables (here $5$) and the number of clauses (here $3$);
  \item Each other line encodes a clause (a disjunction of literals); e.g., \texttt{1 -5 4 0} corresponds to $x_1 \vee \neg x_5 \vee x_4$. Please note that each line terminates with a $0$ -- that is not a variable.
\end{itemize}	
		 
	Your first task is to implement an uninformed random work in Python to find a satisfying assignment of a given CNF in DIMACS format.
	For satisfiable formulas, the output of your program should include lines of the following form:
	
	\begin{verbatim}
	s SATISFIABLE
	v 1 2 3 -4 5
	\end{verbatim}
	
	The first line indicates that the given CNF is satisfiable.
	The second line encodes a satisfying assignment, in this case $x_1 \to True$, $x_2 \to True$, $x_3 \to True$, $x_4 \to False$, $x_5 \to True$. Given this assignment, each clause (/line) of the given CNF should be satisfied.
	
	It should be possible to call your program as \texttt{python solver.py inst1.cnf}, replacing \texttt{solver.py} with your program's name.

	\titledquestion{Solving SAT with SLS}[25]
	
	Your second task is to extend your program by adding two alternative SLS methods of your choice (which we discussed in the lecture) to solve a given CNF. Furthermore, you have to measure the runtime of your three methods on the provided example CNFs to identify the best approach. Discuss the results briefly (e.g., why does approach A perform better than approach B?).	

	\titledquestion{Solving CSP}[10]
		Consider the problem of $N$-Queens, i.e., placing $N$ queens on a chess board of size $N \times N$ such that they cannot capture each other.
		The $4$-Queens problem can be encoded as a CSP in the following JSON format\footnote{See \url{http://bach.istc.kobe-u.ac.jp/sugar/package/current/docs/syntax.html} for a complete format description in a plain text format.}:
		
		\begin{verbatim}
[
 ["int","q_1",1,4],
 ["int","q_2",1,4],
 ["int","q_3",1,4],
 ["int","q_4",1,4],
 ["alldifferent","q_1","q_2","q_3","q_4"],
 ["alldifferent",["+","q_1",1],["+","q_2",2],["+","q_3",3],["+","q_4",4]],
 ["alldifferent",["-","q_1",1],["-","q_2",2],["-","q_3",3],["-","q_4",4]]
]
		\end{verbatim}

		The first $4$ lines encode $4$ variables ($q_1, q_2, q_3, q_4$) with domains $\{1,2,3,4\}$.
		The last $3$ lines encode the constraints as \emph{alldifferent} constraint, i.e., the values of the variables have to differ:
		\begin{itemize}
		  \item $q_1 \neq q_2 \neq q_3 \neq q_4$
		  \item $(q_1 + 1)  \neq (q_2 + 2) \neq (q_3 + 3) \neq (q_4 + 4)$
		  \item $(q_1 - 1)  \neq (q_2 - 2) \neq (q_3 - 3) \neq (q_4 - 4)$
		\end{itemize}
		Please note that the arithmetic terms use a prefix-notation.
		
		Your task is to write a \emph{Python} program (Python version $2.7$ or newer) that
		\begin{itemize}
		  \item reads and interpretes a CSP problem as an input,\\ e.g., \texttt{\$ python csp\_solver.py queens.csp}\\
		  The program should be able to read any CSP in such a format which consists of integer variables, alldifferent constraints and arithmetic terms with operators \texttt{+, -, *, /, \%} (without nesting of terms) -- your program does not have to support all other constructs that are in the full format specification.\\[1em]
		  To read the instance you can simply use the \texttt{json} package:
		\begin{verbatim}
import sys
import json
with open(sys.argv[1]) as fp:
    csp = json.load(fp)
		\end{verbatim}
		  \item solves this problem with one of your implemented SLS approaches from Task II;
		  \item should stop when it found a solution and output it in the following format:
 		\begin{verbatim}
q_1 = 2
q_2 = 4
q_3 = 1
q_4 = 3
		\end{verbatim}
		\end{itemize}

\titledquestion{Feedback}[Bonus: 5]
For each question in this assignment, state:
\begin{itemize}
	\item How long you worked on it.
	\item What you learned.
	\item Anything you would improve in this question if you were teaching the course.
\end{itemize}

\end{questions}


% {\bf This assignment is due on 16.11.2015 (23:59 GMT).} Submit your solution for the tasks by uploading a PDF to our ILIAS\footnote{ \url{https://ilias.uni-freiburg.de/goto.php?target=crs_465155&client_id=unifreiburg}.} course page. The PDF has to include the name of the submitter(s). Teams of at most $2$ students are allowed. Everyone has to submit his/her solution.\\

% Please note that this assignment is optional and you can get bonus points. However, we strongly encourage you to solve the given tasks since we will solve such CSP problems in the next exercise and you will benefit from some experience with modelling and understanding CSP problems. 

\noindent
{\bf This assignment is due on 16.11.2015 (23:59 GMT).} Submit your solution for the tasks by uploading an archive (tar.gz) to our ILIAS\footnote{ \url{https://ilias.uni-freiburg.de/goto.php?target=crs_465155&client_id=unifreiburg}.} course page. The archive has to include the name of the submitter(s). Teams of at most $2$ students are allowed. Everyone has to submit his/her solution.  \textbf{Your scripts should be commented to be readable for the tutors.} Executing them should produce the requested output.


\end{document}