\documentclass{exam}
\usepackage{amsmath, amsfonts}
\usepackage{verbatim}
\usepackage{graphicx}


\usepackage[hyperfootnotes=false]{hyperref}

\usepackage[usenames,dvipsnames]{color}
\newcommand{\note}[1]{
	\noindent~\\
	\vspace{0.25cm}
	\fcolorbox{Red}{Orange}{\parbox{0.99\textwidth}{#1\\}}
	%{\parbox{0.99\textwidth}{#1\\}}
	\vspace{0.25cm}
}
\renewcommand{\note}[1]{}

\newcommand{\hide}[1]{}
\renewcommand{\hide}[1]{#1}

\renewcommand{\vec}[1]{\mathbf{#1}}
\DeclareMathOperator*{\argmin}{argmin}

\qformat{\thequestion. \textbf{\thequestiontitle}\hfill[\thepoints]}
\bonusqformat{\thequestion. \textbf{\thequestiontitle}\hfill[\thepoints]}

\pagestyle{headandfoot}
\firstpageheader{Due: 09.11.2015 (23:59 GMT)}{ {\bf MLOAD} \\ Second Assignment}{M. Lindauer \& F. Hutter\\ WS 2015/16}
\runningheader{Due: 09.11.2015 (23:59 GMT)}{Second Assignment}{WS 2015/16}
\runningfooter{}{}{}
\headrule
\pointsinrightmargin
\bracketedpoints
\marginpointname{pt.}


\begin{document}

%After you now know some examples of hard combinatorial problems, the goal of this second exercise is to let you encode a real world problem as a CSP and to implement a first very simple Python program to read a CSP and solve it.

After you now know some examples of hard combinatorial problems, the goal of this second exercise is to let you gain hands-on knowledge in encoding problems as a CSP. In future exercises, you will write solvers to tackle such problems and optimize such solvers.

\begin{questions}

\titledquestion{CSP Modelling: TSP}[Bonus: 10]

Consider the decision version of the traveling salesperson problem (TSP): Given a set of $n$ cities with a set of costs for getting from any city to any other (a fully connected directed graph with arc costs), is there a tour that visits every city exactly once before returning to the starting city, and that has a total cost less than some specified $k$?

Give two different representations of TSP as a CSP (that is, two representations that use different sets of variables, different sets of constraints or both).  Remember, to specify a CSP representation you need variables, domains and constraints.

\hide{

\begin{tabular}{|l|l|}
\hline
CSP feature &Representation\\
\hline\hline
Variables   &The variables represent where the salesman goes to at each stage of the\\
        &tour.\\
\hline
Domains &The domain of each variable is the set of cities.\\
\hline
Constraints &There is a constraint between every pair of variables, that no two take\\
        &the same value.  There is also a global constraint that the sum of the\\
        &costs doesn't exceed $k$.\\
\hline
\end{tabular}

\begin{tabular}{|l|l|}
\hline
CSP feature &Representation\\
\hline\hline
Variables   &The variables are the possible arcs the agent could follow (from one \\
        &city to the next).\\
\hline
Domains &The domain of each variable is true or false: whether or not the arc is\\
        &part of the tour.\\
\hline
Constraints &There is a constraint between every pair of variables representing arcs\\
        &leading to the same city, that only 1 is used.  There is a similar set of\\
        &constraints between every variable representing arcs leading from the\\
        &same city.  These two sets restrict the agent to visiting every city\\
        &exactly once.  There is a global constraint that the tour be a single loop\\
        &(rather than multiple partial tours).  There is also a global constraint\\
        &that the sum of the costs doesn't exceed $k$.\\
\hline
\end{tabular}
}



	\titledquestion{CSP Modelling: Soccer Scheduling}[Bonus: 10]
	
Consider the problem of designing a schedule for a Soccer league with eight teams.
Games happen every Sunday for seven weeks in a row.
Each team must play each other team once; at least three of each team's game must be home games and at least three must be away games. 
No more than two games in a row may be home games.
The SC Freiburg team (team \#1) must play its first game at home and the VfB Stuttgart team (team \#8) must play its last game at home. 
Use the following variables: 
\begin{itemize}
  \item $p(i,t) \in \{1, \ldots, 8\}$ indicates which team is played by team $i$ in week $t$;
  \item $h(i,t) \in \{0, 1\}$ indicates whether team $i$ plays at home in week $t$.
\end{itemize}
 
Describe the constraints needed to represent this problem as a CSP.
Constraints should be specified as logical tests (i.e., you do not need to enumerate them); 
if you want to indicate that a constraint applies to all teams (for example), 
you can write ``for all $i$: \texttt{<constraint involving $i$>}''.

\hide{
Solution:
\begin{enumerate}
  \item for all $i$, for all $t$, $p(i,t) \neq i$. (``teams never play themselves.)
  \item for all $i$, for all $t' \neq t$, $p(i,t') \neq p(i,t)$. (``team $i$ doesn't play the same team twice.'')
  \item for all $i$,  $p(i,t) = p(p(i,t),t) = i$. (``if $i$ plays $j$, then $j$ plays $i$.'')
  \item $h(i,t) = 1 - h(p(i,t),t)$.  (``if $i$ is at home then their opponent is away, and vice versa.'')
  \item for all $t \in \{3, \ldots, 8\}$, if $h(i,t-1) = 1$ and $h(i,t-2) = 1$ then $h(i,t) = 0$. (``no more than two home games in a row for $i$.'')
  \item for all $i$, $3 \leq \sum_{k=1}^8 h(i,k) \leq 4$. (``number of home games for $i$ is between 3 and 4''.)
 \item $h_{1,1} = 1$. (``Freiburg's first game is at home'')
 \item $h_{8,7} = 1$. (``Stuttgart's last game is at home.'')
\end{enumerate}
}
		
		
		
% \hide{		
% 	\titledquestion{Solving CSP with uninformed random work (URW)}[32]
% 		Consider the problem of $N$-Queens, i.e., placing $N$ queens on a chess board of size $N \times N$ such that they cannot capture each other.
% 		The $4$-Queens problem can be encoded as a CSP in the following JSON format\footnote{See \url{http://bach.istc.kobe-u.ac.jp/sugar/package/current/docs/syntax.html} for a complete format description in a plain text format}:
% 		
% 		\begin{verbatim}
% [
%  ["int","q_1",1,4],
%  ["int","q_2",1,4],
%  ["int","q_3",1,4],
%  ["int","q_4",1,4],
%  ["alldifferent","q_1","q_2","q_3","q_4"],
%  ["alldifferent",["+","q_1",1],["+","q_2",2],["+","q_3",3],["+","q_4",4]],
%  ["alldifferent",["-","q_1",1],["-","q_2",2],["-","q_3",3],["-","q_4",4]]
% ]
% 		\end{verbatim}
% 
% 		The first $4$ lines encode $4$ variables ($q_1, q_2, q_3, q_4$) with domains $\{1,2,3,4\}$.
% 		The last $3$ lines encode the constraints as \emph{alldifferent} constraint, i.e., the values of the variables have to differ:
% 		\begin{itemize}
% 		  \item $q_1 \neq q_2 \neq q_3 \neq q_4$
% 		  \item $(q_1 + 1)  \neq (q_2 + 2) \neq (q_3 + 3) \neq (q_4 + 4)$
% 		  \item $(q_1 - 1)  \neq (q_2 - 2) \neq (q_3 - 3) \neq (q_4 - 4)$
% 		\end{itemize}
% 		Please note that the arithmetic terms use a prefix-notation.
% 		
% 		Your task is to write a \emph{Python} program (Python version $2.7$ or newer) that
% 		\begin{itemize}
% 		  \item reads and interpretes a CSP problem as an input,\\ e.g., \texttt{\$ python csp\_solver.py queens.csp}\\
% 		  The program should be able to read any CSP in such a format which consists of integer variables, alldifferent constraints and arithmetic terms with operators \texttt{+, -, *, /, \%} (without nesting of terms) -- your program does not have to support all other constructs that are in the full format specification; \newpage
% 		  To read the instance you can simply use the \texttt{json} package:
% 		\begin{verbatim}
% import sys
% import json
% with open(sys.argv[1]) as fp:
%     csp = json.load(fp)
% 		\end{verbatim}
% 		  \item solves this problem with an uninformed random walk (URW);
% 		  \item should stop when it found a solution and output it in the following format:
%  		\begin{verbatim}
% q_1 = 2
% q_2 = 4
% q_3 = 1
% q_4 = 3
% 		\end{verbatim}
% 		\end{itemize}
% }
% 	
\end{questions}

\noindent
{\bf This assignment is due on 09.11.2015 (23:59 GMT).} Submit your solution for the tasks by uploading a PDF to our ILIAS\footnote{ \url{https://ilias.uni-freiburg.de/goto.php?target=crs_465155&client_id=unifreiburg}.} course page. The PDF has to include the name of the submitter(s). Teams of at most $2$ students are allowed. Everyone has to submit his/her solution.\\

Please note that this assignment is optional and you can get bonus points. However, we strongly encourage you to solve the given tasks since we will solve such CSP problems in the next exercise and you will benefit from some experience with modelling and understanding CSP problems. 

%{\bf This assignment is due on 09.11.2015 (23:59 GMT).} Submit your solution for the tasks by uploading an archive (tar.gz) to our ILIAS\footnote{ \url{https://ilias.uni-freiburg.de/goto.php?target=crs_465155&client_id=unifreiburg}.} course page. The archive has to include the name of the submitter(s). Teams of at most $2$ students are allowed. Everyone has to submit his/her solution.  \textbf{Your scripts should be commented to be readable for the tutors.} Executing them should produce the requested output.


\end{document}