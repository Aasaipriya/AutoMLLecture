\documentclass{exam}
\usepackage{amsmath,amssymb,amsthm,mathrsfs,amsfonts,dsfont}
\usepackage{verbatim}
\usepackage{graphicx}


\usepackage[hyperfootnotes=false]{hyperref}

\usepackage[usenames,dvipsnames]{color}
\newcommand{\note}[1]{
	\noindent~\\
	\vspace{0.25cm}
	\fcolorbox{Red}{Orange}{\parbox{0.99\textwidth}{#1\\}}
	%{\parbox{0.99\textwidth}{#1\\}}
	\vspace{0.25cm}
}
%\renewcommand{\note}[1]{}
\newcommand{\hide}[1]{#1}
\renewcommand{\hide}[1]{}

\renewcommand{\vec}[1]{\mathbf{#1}}
\DeclareMathOperator*{\argmin}{argmin}

\qformat{\thequestion. \textbf{\thequestiontitle}\hfill[\thepoints]}
\bonusqformat{\thequestion. \textbf{\thequestiontitle}\hfill[\thepoints]}

\pagestyle{headandfoot}
\firstpageheader{Due: 29.02.2016 (23:59 GMT)}{ {\bf MLOAD} \\ Final Project}{M. Lindauer \& F. Hutter\\ WS 2015/16}
\runningheader{Due: 29.02.2016 (23:59 GMT)}{Final Project}{WS 2015/16}
\runningfooter{}{}{}
\headrule
\pointsinrightmargin
\bracketedpoints
\marginpointname{pt.}


\begin{document}

This final project is necessary for the oral exam. 
The purpose of this project is that you get hands-on experience on (nearly) all topics of the course 
and you can present and explain the results of your work. 
To this end, please submit your code, plots ([$\to$ PDF]), tables ([$\to$ PDF]) etc. to our ILIAS webpage.
In the first $15$ minutes of the exam, you will have the chance to explain your approach and the results.
Please do not prepare any presentation slides -- it will be a relaxed and interactive conversation!
  
\begin{questions}

\titledquestion{Performance Optimization of \textit{SAPS}}[-]

\emph{Your final task is to optimize the performance of the SAT solver SAPS on the provided instances.}
How you optimize the performance of \textit{SAPS} is up to you. 
You could use algorithm configuration and/or algorithm selection.\footnote{Since we provide an already implemented SAT solver, hyper-heuristics are not straightforward to apply.} 
In the end, you should convince us that you indeed optimized the performance of SAPS.
To this end, you should think about the following tasks:

\begin{itemize}
  \item Measure the default performance of \textit{SAPS};
  \item Plot the performance (CDF and boxplots of the RTD);
  \item Guess whether it is a heterogeneous or an homogeneous instance set;
  \item Apply algorithm configuration to determine a well-performing configuration;
  \item Determine the importance of \textit{SAPS}'s parameters;
  \item Optimize an algorithm schedule of configurations of \textit{SAPS};
  \item Apply algorithm selection to select well-performing configurations of \textit{SAPS};
  \item Plot the performance of the optimized \textit{SAPS} vs the default \textit{SAPS} (e.g., scatter plot); 
  \item Report the performance (PAR1, PAR10, number of timeouts).
\end{itemize}

Please note that you do not necessarily have to apply all these methods -- pick the ones that you think are most appropriate.

We provide the following files and binaries:

\begin{itemize}
  \item An instance set consisting of quasigroup completion problems (QCP) and graph colouring problems (SWGCP) -- please note that the instance files have a size of nearly $15$GB.
  \item A file listing all test instances -- to compare your solutions against the ones of your colleagues, you can upload your performance\footnote{\url{https://docs.google.com/spreadsheets/d/1gyKBLlhrHd_uLRzLO9uWXWkj_ot38OdbsZVvWhHYYVg/}} 
  \item The binary of \textit{SAPS} (call: \texttt{ubcsat -alg saps}) with parameters \texttt{alpha}, \texttt{rho}, \texttt{ps}, \texttt{wp}
  \item A binary to generate instance features for a SAT formula (binary: \texttt{features}) -- We recommend to use the option \texttt{-base} to get a cheap set of features
\end{itemize}

%\note{ML: unfortunately, the parameters of SAPS are not documented in the help.}
 
Furthermore, you can of course use all scripts and tools you already know from the exercises; however, you are not limited to them.

You should respect the following constraints:

\begin{itemize}
  \item The performance has to be measured in terms of search steps (as a CPU-independent cost metric).\footnote{If you apply algorithm configuration, you have to return the number of search steps as solution cost/quality (reps. $10$ times the maximal number o search steps for timeouts.}
  \item \textit{SAPS} should not run longer than $100\ 000$ search steps on each instance.
  \item Use the option \texttt{-r satcomp} to get the already known output format of the SAT competition.
\end{itemize}

\end{questions}

\noindent
{\bf This assignment is due on 29.02.2016 (23:59 GMT).} Submit your solution for the tasks by uploading an archive (tar.gz) to our ILIAS\footnote{ \url{https://ilias.uni-freiburg.de/goto.php?target=crs_465155&client_id=unifreiburg}.} course page. The archive has to include the name of the submitter. \textbf{Teams are not allowed. 
However, we encourage you to discuss your approaches in our ILIAS forum and to ask questions there if you have any -- code distribution is not allowed.}

\bigskip
\paragraph{General constraints for code submissions}

\begin{itemize}
  %\item The program can be called as stated on the exercise sheet.
  %\item The program exactly returns the required output (neither less nor more) -- please use a \texttt{--verbose} option to increase the verbosity level for debugging.
  \item Your scripts should be commented to be readable for the tutors. All functions and classes are documented with a docstring. 
  \item Provide a README ($\to$ how to install requirements and run your program(s)) and (if necessary) an installation script if your program requires any other packages.
\end{itemize}

\end{document}