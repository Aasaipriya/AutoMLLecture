\documentclass{exam}
\usepackage{amsmath, amsfonts}
\usepackage{verbatim}
\usepackage{graphicx}


\usepackage[hyperfootnotes=false]{hyperref}

\usepackage[usenames,dvipsnames]{color}
\newcommand{\note}[1]{
	\noindent~\\
	\vspace{0.25cm}
	\fcolorbox{Red}{Orange}{\parbox{0.99\textwidth}{#1\\}}
	%{\parbox{0.99\textwidth}{#1\\}}
	\vspace{0.25cm}
}
\renewcommand{\note}[1]{}

\renewcommand{\vec}[1]{\mathbf{#1}}
\DeclareMathOperator*{\argmin}{argmin}

\qformat{\thequestion. \textbf{\thequestiontitle}\hfill[\thepoints]}
\bonusqformat{\thequestion. \textbf{\thequestiontitle}\hfill[\thepoints]}

\pagestyle{headandfoot}
\firstpageheader{Due: 02.11.2015 (23:59 GMT)}{ {\bf MLOAD} \\ First Assignment}{M. Lindauer \& F. Hutter\\ WS 2015/16}
\runningheader{Due: 02.11.2015 (23:59 GMT)}{First Assignment}{WS 2015/16}
\runningfooter{}{}{}
\headrule
\pointsinrightmargin
\bracketedpoints
\marginpointname{pt.}


\begin{document}

\note{Careful: ``Algorithm design'' is obviously what everyone does. ``Automated''!}
The automated algorithm design methods you will learn about in this course help you improve the empirical performance of algorithms.
The goal of this first exercise is to get you thinking about possible applications of these methods (in particular, \emph{algorithm configuration} methods) to algorithms you know about from previous experience.

\begin{questions}
	\titledquestion{Algorithms with tunable parameters}[20]
	\note{We should in general avoid to say ``Parameterized algorithms'' since too many people confuse it with ``Parameterized complexity''. Rather say ``algorithms with parameters''. ``Highly-parameterized algorithms'' is also OK.}
		Identify $2$ algorithms that have tunable parameters. Describe each algorithm briefly (without code!) and explain its parameters; this includes:
		\begin{itemize}
		  \item Why does the parameter influence the performance of the algorithm? 
		  \item What could be a good range of possible values of the parameter (e.g., $[0,1]$, $[1,1024]$ or $\{yes,no\}$)? Briefly explain your choices.
		\end{itemize}
		
	\titledquestion{Manual Optimization}[20]
		If you would have to optimize the parameters of your algorithm, how would you proceed? Describe your approach briefly (maybe use pseudo-code) and analyze the effort you would need to apply this approach to one of your algorithms from Task 1. 
		
		For example, you have an algorithm with three tunable parameters and each of them has four possible values. So, there are $4^3$ possible combinations of parameter values. Let us assume that each algorithm run needs 
		%at most 
		$10$ CPU seconds and you have $50$ problem instances. 
		If we were to assess the performance of all possible parameter combinations on all instances, we would need 
		\note{it was ``at most'' before and here ``at least''; that's inconsistent; I dropped both.}
		%at least 
		$4^3 \cdot 10 \cdot 50 = 32000$ CPU seconds to determine the best possible configuration. 
		\emph{Your approach has to be faster than this brute force method; it is OK if it not guaranteed to always yield the perfect solution.}
		\note{I added the part about no performance guarantee being necessary since this problem otherwise does not have a general solution.}
		%Note that you should think of a better way than trying all possible parameter configurations.}
	
	\titledquestion{Complexity Analysis}[10]
		As a preparation for the next lecture, analyze the computational complexity of one of your algorithms from Task 1 in big-O notation -- for example, $O(n^3)$ or $O(2^n)$. 
		Does your algorithm solve an NP-complete problem? What's the best known complexity for solving the problem your algorithm solves?
\end{questions}

\noindent
{\bf This assignment is due on 02.11.2015 (23:59 GMT).} Submit your solution for the tasks by uploading a PDF to our ILIAS\footnote{ \url{https://ilias.uni-freiburg.de/goto.php?target=crs_465155&client_id=unifreiburg}.} course page. The PDF has to include the name of the submitter(s). Teams of at most $2$ students are allowed. Everyone has to submit his/her solution.  %Your scripts should be commented to be readable for the tutors. Executing them should produce the requested plots either as files.

\end{document}