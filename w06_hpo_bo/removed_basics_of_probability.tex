\section{Bayesian Optimization}
%----------------------------------------------------------------------
%----------------------------------------------------------------------
\begin{frame}[c]{Bayesian Optimization: Conditional probability}

\begin{block}{Conditional probability - definition}
    Let $A$ and $B$ be two events with $P(B) \neq 0$. The conditional probability of $A$ given $B$ is defined to be:
	\begin{equation}
	    P(A \vert B) = \frac{P(A \cap B)}{P(B)}
    \label{eq:cond_prob}  
	\end{equation}
\end{block}

\pause

\begin{block}{Conditional probability - example}
   You toss a fair coin three times. Given that you have observed at least one heads, what is the probability that you observe at least two heads? 
\end{block}

\pause

\begin{block}{Conditional probability - solution}
	\begin{equation*}
    \begin{aligned}
        A_1 = S - \{TTT\}, \pause \textrm{ and } A_2 = \{HHT, HTH ,THH, HHH\} \\ \pause
        P(A_2 \vert A_1) =  \pause \frac{P(A_2 \cap A_1)}{P(A_1)} =  \pause \frac{P(A_2)}{P(A_1)} =  \pause \frac{4}{7}
    \end{aligned}
    \end{equation*}
\end{block}

\note[item]{source: https://www.probabilitycourse.com/chapter1/1\_4\_5\_solved3.php}

\note[item]{S - all possibilities}

\note[item]{Let $A_1$ be the event that you observe at least one heads, and $A_2$ be the event that you observe at least two heads.}

\end{frame}
%-----------------------------------------------------------------------

%----------------------------------------------------------------------
%----------------------------------------------------------------------
\begin{frame}[c]{Bayesian Optimization: Bayes rule}

\begin{itemize}
    \item Using the definition of conditional probability (Eq. \ref{eq:cond_prob}), one can rearrange the terms to show:
        \begin{equation*}
            P(A \cap B) = P(A \vert B) * P(B)
        \end{equation*}
    \item Similarly, it follows:
        \begin{equation*}
            P(B \cap A) = P(B \vert A) * P(A)
        \end{equation*}
        
        \begin{block}{Bayes rule (theorem)}
        Since $A \cap B = B \cap A$, one can rewrite both above relations as:
        	\begin{equation}
        	    P(A \vert B) = \frac{P(B \vert A) * P(A)}{P(B)}
                \label{eq:bayes_rule}  
        	\end{equation}
        \end{block}

\end{itemize}
\end{frame}
%-----------------------------------------------------------------------

%----------------------------------------------------------------------
%----------------------------------------------------------------------
\begin{frame}[c]{Bayesian Optimization: Bayes rule - example}

\begin{block}{Bayes rule - example}
    You are planning a picnic today, but the morning is cloudy: \pause
    \begin{itemize}
        \item 50\% of all rainy days start off cloudy, \pause
        \item cloudy mornings are common (about 40\% of days start cloudy), \pause
        \item it is a dry month (only 3 of 30 days tend to be rainy, or 10\%). \pause
    \end{itemize}
    
    \emph{What is the chance of rain during the day?}
\end{block}

 \pause

\begin{block}{Bayes rule - solution}
	\begin{equation*}
	\begin{aligned}
	    P(RainyDay \vert CloudyMorning) = \frac{P(CloudyMorning \vert RainyDay) * P(RainyDay)}{P(CloudyMorning)} \\  \pause
	    P(RainyDay \vert CloudyMorning) = \frac{0.5 * 0.1}{0.4} = 0.125
	\end{aligned}
	\end{equation*}
\end{block}

\note[item]{source: https://www.mathsisfun.com/data/bayes-theorem.html}
\note[item]{https://www.countbayesie.com/blog/2015/2/18/bayes-theorem-with-lego}
        
\end{frame}
%-----------------------------------------------------------------------

%----------------------------------------------------------------------

%-----------------------------------------------------------------------
