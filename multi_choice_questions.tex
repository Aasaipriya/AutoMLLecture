\documentclass{exam}
\usepackage[utf8]{inputenc}
\usepackage{multirow}

%-------------------------------------------------------------
\title{Collection of AutoML multiple Choice Questions}

\begin{document}
\maketitle
%-------------------------------------------------------------------
\section{Big Picture}
%Here, the questions begin
All questions require a single choice answer.
\begin{questions}

\question AutoML can help you to ...
\begin{choices}
 \choice replace the job of someone by a machine.
 \choice enables efficient interdisciplinary projects.
 \choice increase the time for finishing ML projects.
 \choice apply ML without any knowledge about the risks of ML.
\end{choices}

\question AutoML is a hard problem because ... (among other reasons).
\begin{choices}
 \choice every function evaluation is super cheap
 \choice we only have to tune a single hyperparameter
 \choice we need the same ML pipeline for each dataset
 \choice the search space can quite complex and large
\end{choices}

\question The CASH problem stands for ...
\begin{choices}
 \choice Combined Algorithm Solving and Hashing.
 \choice Combined Algorithm Selection and Hyperparameter Optimization.
 \choice Combined Automated Selection of Heuristics.
 \choice Combined Automated Selection of Hyperparameters.
\end{choices}

\question  To apply AutoML to deep learning, we ...
\begin{choices}
 \choice want to ultimately optimize in the joint space of architectures and hyperparameters.
 \choice first optimize the architectures and afterwards the hyperparameters.
 \choice irst optimize the hyperparameters and afterwards the architecture.
\end{choices}

\question In dynamic algorithm configuration, we ...
\begin{choices}
 \choice select a configuration specifically to each instance at hand.
 \choice predict which configuration an algorithm should use.
 \choice search for fixed dynamic schedule of configurations.
 \choice learn a policy that predicts a configuration for a given algorithm state.
\end{choices}

\question To mitigate the risks of AutoML, we ...
\begin{choices}
 \choice will build systems that will avoid all of these risks automatically.
 \choice will blame ML and not AutoML.
 \choice teach others about potential risks of ML and AutoML.
 \choice will never publish our code as open-source.
\end{choices}

%-------------------------------------------------------------------------------
\section{Evaluation}
\question Training Machine Learning Models is ...
\begin{choices}
    \choice trivially easy.
    \choice fundamentally an optimization problem. % Correct
    \choice always cheap.
\end{choices}

\question Performance estimation of a model = performance estimation of an algorithm
\begin{choices}
    \choice True
    \choice False % Correct
\end{choices}

\question Reason for bad learning curves include ...
\begin{choices}
    \choice underfitting % Correct
    \choice overfitting % Correct
    \choice optimal hyperparameter settings
    \choice high bias   % Correct
\end{choices}

\question Choose all that apply:
\begin{choices}
    \choice underfitting $\mapsto$ high bias in model % Correct
    \choice underfitting $\mapsto$ high variance in model
    \choice overfitting $\mapsto$ high bias in model
    \choice overfitting $\mapsto$ high variance in model % Correct
\end{choices}

\question In a statistical hypothesis test with $\alpha = 0.05$ and $p < \alpha$ we
\begin{choices}
    \choice Reject $H_0$ % Correct
    \choice Accept $H_0$
    \choice We cannot draw a conclusion.
\end{choices}

\question In a statistical hypothesis test with $\alpha = 0.05$ and $p > \alpha$ we
\begin{choices}
    \choice Reject $H_0$
    \choice Accept $H_0$
    \choice We cannot draw a conclusion. % Correct
\end{choices}

\question Is it permissible to first look at your $p$ value before choosing your $\alpha$?
\begin{choices}
    \choice Yes
    \choice No % Correct
\end{choices}

\question For two algorithms X and Y we list their classification performance in the following table, with $0$ indicating a wrong classification and $1$ indicating correct classification.
    \begin{center}
      \begin{tabular}{cc|cc}
          & & \multicolumn{2}{c}{X} \\
          & & $0$ & $1$ \\
          \hline
          \multirow{2}{*}{Y} & $0$ & 30 & 90 \\
          & $1$ & 75 & 5 \\
      \end{tabular}
    \end{center}
    Use the McNemar Test where the \texttt{$H_0$: both models have the same performance} and \texttt{$H_1$: performances of the models are not equal}  with $\alpha = 0.05$\footnote{$\chi^2_1 = 3.841$}
    % Chi^2 = 1.188
\begin{choices}
    \choice Reject $H_0$
    \choice Accept $H_0$
    \choice We cannot draw a conclusion. % Correct
\end{choices}

\question Instead consider now a different performance table with the rest being equal.
    \begin{center}
      \begin{tabular}{cc|cc}
          & & \multicolumn{2}{c}{X} \\
          & & $0$ & $1$ \\
          \hline
          \multirow{2}{*}{Y} & $0$ & 30 & 3 \\
          & $1$ & 17 & 5 \\
      \end{tabular}
    \end{center}
    % 17 + 3 <= 20 -> We shouldn't use McNemar
\begin{choices}
    \choice Reject $H_0$
    \choice Accept $H_0$
    \choice We cannot draw a conclusion.
    \choice We shouldn't use McNemar % Correct
\end{choices}

\question The post-hoc Nemenyi test ...
\begin{choices}
    \choice should be used before the Friedman test.
    \choice is used to determine if algorithms have the same ranks.
    \choice compares all pairs of algorithms to find best-performing algorithm after $H_0$ of the
Friedman-test was rejected. % Correct
\choice can not be used with more than 2 algorithms.
\end{choices}
\end{questions}
\end{document}
